%%%%%%%%%%%%%%%%%%%%%%%%%%%%%%%%%%%%%%%%%
% Journal Article
% LaTeX Template
% Version 1.4 (15/5/16)
%
% This template has been downloaded from:
% http://www.LaTeXTemplates.com
%
% Original author:
% Frits Wenneker (http://www.howtotex.com) with extensive modifications by
% Vel (vel@LaTeXTemplates.com)
%
% License:
% CC BY-NC-SA 3.0 (http://creativecommons.org/licenses/by-nc-sa/3.0/)
%
%%%%%%%%%%%%%%%%%%%%%%%%%%%%%%%%%%%%%%%%%

%----------------------------------------------------------------------------------------
%	PACKAGES AND OTHER DOCUMENT CONFIGURATIONS
%----------------------------------------------------------------------------------------

\documentclass[twoside,twocolumn]{article}

\usepackage{blindtext} % Package to generate dummy text throughout this template 
%\usepackage[utf8]{inputenc} % Package for unicode characters
\usepackage[utf8]{inputenc}
\usepackage{amssymb}
\usepackage{newunicodechar}
\newunicodechar{Ɖ}{\DH}

\usepackage[sc]{mathpazo} % Use the Palatino font
\usepackage[T1]{fontenc} % Use 8-bit encoding that has 256 glyphs
\linespread{1.05} % Line spacing - Palatino needs more space between lines
\usepackage{microtype} % Slightly tweak font spacing for aesthetics

\usepackage[english]{babel} % Language hyphenation and typographical rules

\usepackage[hmarginratio=1:1,top=32mm,columnsep=20pt]{geometry} % Document margins
\usepackage[hang, small,labelfont=bf,up,textfont=it,up]{caption} % Custom captions under/above floats in tables or figures
\usepackage{booktabs} % Horizontal rules in tables

\usepackage{lettrine} % The lettrine is the first enlarged letter at the beginning of the text

\usepackage{enumitem} % Customized lists
\setlist[itemize]{noitemsep} % Make itemize lists more compact

\usepackage{abstract} % Allows abstract customization
\renewcommand{\abstractnamefont}{\normalfont\bfseries} % Set the "Abstract" text to bold
\renewcommand{\abstracttextfont}{\normalfont\small\itshape} % Set the abstract itself to small italic text

\usepackage{titlesec} % Allows customization of titles
\renewcommand\thesection{\Roman{section}} % Roman numerals for the sections
\renewcommand\thesubsection{\roman{subsection}} % roman numerals for subsections
\titleformat{\section}[block]{\large\scshape\centering}{\thesection.}{1em}{} % Change the look of the section titles
\titleformat{\subsection}[block]{\large}{\thesubsection.}{1em}{} % Change the look of the section titles

\usepackage{fancyhdr} % Headers and footers
\pagestyle{fancy} % All pages have headers and footers
\fancyhead{} % Blank out the default header
\fancyfoot{} % Blank out the default footer
\fancyhead[C]{The Blockchain Alliance For Good $\bullet$ April 2017 $\bullet$
Vol.
I, No.
1} % Custom header text
\fancyfoot[RO,LE]{\thepage} % Custom footer text

\usepackage{titling} % Customizing the title section

\usepackage[pagebackref]{hyperref} % For hyperlinks in the PDF

%----------------------------------------------------------------------------------------
%	TITLE SECTION
%----------------------------------------------------------------------------------------
\setlength{\droptitle}{-4\baselineskip} % Move the title up

\pretitle{\begin{center}\Huge\bfseries} % Article title formatting
\posttitle{\end{center}} % Article title closing formatting
\title{The Blockchain Manifesto} % Article title
\author{%
\textsc{Daniel Kaminski de Souza}\thanks{The author would like to thank the
Ethereum Classic community.} \\[1ex] % Your name
\normalsize University of Nicosia \\ % Your institution
\normalsize
\href{mailto:daniel.souza@cceg.org.uk}{daniel.souza@cceg.org.uk}
% Your email address
\and % Uncomment if 2 authors are required, duplicate these 4 lines if more
%\textsc{Olinga Taeed}\thanks{Corresponding author} \\[1ex] % Second author's
\textsc{Maryam Taghiyeva}\thanks{} \\[1ex] % Second author's name
\normalsize University of Northampton \\ % Second author's institution
\normalsize \href{mailto:maryam.taghiyeva@cceg.org.uk}{maryam.taghiyeva@cceg.org.uk}
% Second author's email address
\and % Uncomment if 2 authors are required, duplicate these 4 lines if more
%\textsc{Olinga Taeed}\thanks{Corresponding author} \\[1ex] % Second author's
\textsc{Olinga Taeed}\thanks{Director of the \href{www.cceg.org.uk}{Centre for Citizenship, Enterprise
and Governance}. CCEG is a spin out Think Tank from the University of
Northampton.} \\[1ex] % Second author's name
\normalsize University of Northampton \\ % Second author's institution
\normalsize \href{mailto:olinga.taeed@cceg.org.uk}{olinga.taeed@cceg.org.uk} %
% Second author's email address
}
\date{\today} % Leave empty to omit a date
\renewcommand{\maketitlehookd}{%
\begin{abstract}
\noindent These are the principles that guide our mission to promote openness,
innovation \& opportunity on the Blockchain. Based on The Mozilla
Manifesto\cite{MozillaXXXX}.
\end{abstract}
}

%----------------------------------------------------------------------------------------

\begin{document}

% Print the title
\maketitle

%----------------------------------------------------------------------------------------
%	ARTICLE CONTENTS
%----------------------------------------------------------------------------------------

\section{Introduction}

\lettrine[nindent=0em,lines=3]{T}he Blockchain is becoming an increasingly
important part of our lives.

The Blockchain Alliance For Good project is a global community of people who
believe that openness, innovation, and opportunity are key to the continued
health of the Blockchain. We have worked together since 2016 to ensure that the
Blockchain is developed in a way that benefits everyone.

The Blockchain Alliance For Good project uses a community-based approach to
create world-class open source software and to develop new types of
collaborative activities. We create communities of people involved in making the
Blockchain experience better for all of us.

As a result of these efforts, we have distilled a set of principles that we
believe are critical for the Blockchain to continue to benefit the public good
as well as commercial aspects of life. We set out these principles below.

The goals for the Manifesto are to:

\begin{itemize}
  \item articulate a vision for the Blockchain that Alliance participants want
  the Blockchain Alliance For Good to pursue;
  \item speak to people whether or not they have a technical
background;
  \item make Alliance contributors proud of what we're doing and motivate us
to continue;
  \item and provide a framework for other people to advance this vision of
the Blockchain.
\end{itemize}

These principles will not come to life on their own. People are needed to make
the Blockchain open and participatory - people acting as individuals, working
together in groups, and leading others. The Blockchain Alliance For Good is
committed to advancing the principles set out in the Alliance's Manifesto. We
invite others to join us and make the Blockchain an ever better place for
everyone.

\section{Principles}
\begin{itemize}
  \item The Blockchain is an integral part of modern life—a key component in
  education, communication, collaboration, business, entertainment and society
  as a whole.
  \item The Blockchain is a global public resource that must remain open and
  accessible.
  \item The Blockchain must enrich the lives of individual human beings.
  \item Individuals’ security and privacy on the Blockchain are fundamental and
must not be treated as optional.
  \item Individuals must have the ability to shape the Blockchain and their own
experiences on it.
  \item The effectiveness of the Blockchain as a public resource depends upon
interoperability (protocols, data formats, content), innovation and
decentralized participation worldwide.
  \item Free and open source software promotes the development of the Blockchain
as a public resource.
  \item Transparent community-based processes promote participation,
  accountability and trust.
  \item Commercial involvement in the development of the Blockchain brings many
 benefits; a balance between commercial profit and public benefit is critical.
  \item Magnifying the public benefit aspects of the Blockchain is an important
   goal, worthy of time, attention and commitment.
\end{itemize}

\section{Advancing the Blockchain Manifesto}
There are many different ways of advancing the principles of the Alliance's
Manifesto. We welcome a broad range of activities, and anticipate the same
creativity that Alliance participants have shown in other areas of the project.
For individuals not deeply involved in the Blockchain Alliance For Good project,
one basic and very effective way to support the Manifesto is to use Seratio
metrics and other products that embody the principles of the Manifesto.

\section{Blockchain Alliance For Good Pledge}
The Blockchain Alliance For Good pledges to support the Alliance's Manifesto in
its activities. Specifically, we will:

\begin{itemize}
  \item build and enable open-source technologies and communities that support
  the Manifesto’s principles;
  \item build and deliver great consumer products that support the Manifesto’s
  principles;
  \item use the Alliance assets (intellectual property such as copyrights and
  trademarks, infrastructure, funds, and reputation) to keep the Blockchain an
  open platform;
  \item promote models for creating economic value for the public benefit;
  \item and promote the Alliance's Manifesto principles in public discourse and
  within the Blockchain industry.
\end{itemize}

Some Alliance activities—currently the creation, delivery and promotion of
consumer products are conducted primarily through the
\href{www.cceg.org.uk}{Centre for Citizenship, Enterprise and Governance} wholly
owned subsidiary, the CCEG Blockchain UN Lab.

As also happened in the early internet days, the lack of skilled
blockchain developers is also a major concern of the Alliance and is currently
being addressed by strategic partnerships with educational institutions.

\section{Invitation}

The Blockchain Alliance For Good invites all others who support the principles of the
Alliance's Manifesto to join with us, and to find new ways to make this vision
of the Blockchain a reality.

%Maecenas sed ultricies felis. Sed imperdiet dictum arcu a egestas. 
%\begin{itemize}
%\item Donec dolor arcu, rutrum id molestie in, viverra sed diam
%\item Curabitur feugiat
%\item turpis sed auctor facilisis
%\item arcu eros accumsan lorem, at posuere mi diam sit amet tortor
%\item Fusce fermentum, mi sit amet euismod rutrum
%\item sem lorem molestie diam, iaculis aliquet sapien tortor non nisi
%\item Pellentesque bibendum pretium aliquet
%\end{itemize}
%\blindtext % Dummy text

%Text requiring further explanation\footnote{Example footnote}.

%------------------------------------------------

%\section{Results}

%\begin{table}
%\caption{Example table}
%\centering
%\begin{tabular}{llr}
%\toprule
%\multicolumn{2}{c}{Name} \\
%\cmidrule(r){1-2}
%First name & Last Name & Grade \\
%\midrule
%John & Doe & $7.5$ \\
%Richard & Miles & $2$ \\
%\bottomrule
%\end{tabular}
%\end{table}

%\blindtext % Dummy text

%\begin{equation}
%\label{eq:emc}
%e = mc^2
%\end{equation}

%\blindtext % Dummy text

%------------------------------------------------

%\section{Discussion}

%\subsection{Subsection One}

%A statement requiring citation \cite{Figueredo2009}.
%\blindtext % Dummy text

%\subsection{Subsection Two}

%\blindtext % Dummy text

%----------------------------------------------------------------------------------------
%	REFERENCE LIST
%----------------------------------------------------------------------------------------

\bibliographystyle{abbrv}  
\bibliography{tex/bibliography}

%----------------------------------------------------------------------------------------

\end{document}
